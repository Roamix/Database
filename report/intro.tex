\section{Introduction}
% Christoffer

% Introduction to train management:
The database system of this project is a train management schema, which stores 
trains, stations, tracks, routes, employees, etc. So a railway system that 
connects cities and stations is what we are modelling with our \emph{Train 
Management} schema.

This report will go through the process of designing a database, implementing 
it, and extend its functionality by programming functions, triggers etc.
The function of the databse system is to store important information about a 
railway system, such as routes and tracks between cities and stations. With the 
relations in the \emph{Train Management} schema, it is possible to plan a route 
from station $A$ to station $B$ and calculate the shortest route to 
reach the destination. However, these more advanced algorithms, 
\textit{"Breadth-first search"} and \textit{"Dijkstra's Shortest Path"}, to 
plan routes and calculate shortest paths, are not implemented or stored as 
procedures in this schema, as such algorithms are more conveniently implemented 
in objective programming languages. But it would of course be possible to 
implement either of these in any DBMS, such as ODBC or JDBC.

%However that is not say that it would not be possible The \emph{Train 
%Management} schema is however, compatible with the 
%implementation of these algorithm. So by using the correct queries on the 
%relations it is definitely possible to retrieve the required information that 
%these algorithms need, and they could then be implemented in another DBMS like 
%ODBC or JDBC.

Furthermore the problem of finding shortest paths seem redundant for routes as it is not necessarily the shortest paths that are 
most conveniently made into routes, as the network of routes should make it as 
fast as possible to get from a station to another, regardless of where you are 
moving from and to. This problem is easily solved by having routes between 
every pair of stations. However, that is a very cost inefficient solution, as 
it requires $O(n^2)$ routes. Where $n$ is the number of stations. Since we need 
trains on all routes, this quickly becomes a very expensive solution. 

%Database type
The database is then categorised as an Information Database, that provides information about train systems and train routes. From the database it is possible to extract important information about how the railway system is mapped and routed, and it is in compliance with possible graph algorithms.


% What is the purpose? 
	%Describe part of real world to be modelled
% What is this report about?
% What functionality do we want?
	%Be able to have cities and stations be connected by routes and tracks, and store it in our database?
	%Be able to plan routes?
% What are the limitations, i.e. is there something we are not goint to include?
	%Complicated graph algorithms to calculate 		shortest routes etc. are not implemented
	%Could have used BFS and Dijkstra's to make a routeplanning stored procedure.

% What sort of database is it? 
% (Production, Transaction, Fulltext, Temporal or Data Warehouse database)