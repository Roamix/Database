\section{Conclusion}
% Tobias

% Discussion paragraphs first (if any)

% Comment on final result
The goal of the project was to generate a functioning database train 
management. We did this by first designing the database as en Entity Relation 
diagram. Once the diagram was finished it was translated into the MySQL EER 
diagram. However, at this point we realised flaws in the initial design, and an 
iterative process thereby began. One entity or relation was translated possibly 
redesigned translated again. Only once we were satisfied, we moved on to the 
next relation/entity. This gave a nice workflow as each aspect of the database 
was considered in turn.

% Does the implementation follow the design (YES - generated from "sketch")
% Where does it not? Why?
After the design (ER diagram) had been translated to the EER diagram, the 
database could be forward engineered. This had the huge advantage that 
implementation automatically becomes very much like the design, assuming the 
translation did not change too much. By very little effort the ``sketch'' of a 
database was made into an actual database, at which point we could start 
programming functionality.

% Comments on work -process, -flow, and possibly -distribution
% Mentioned briefly above

% Future expansions?

