\section{Implementation} \label{sec:implementation}
% Tobias

% Forward / Backward engineering
When creating the database we used forward engineering to auto generate the 
database from a MySQL EER diagram. The diagram can be seen in 
Figure~\ref{fig:EER}. This had the advantage that the EER diagram could easily 
be made as a translation of the ER Diagram described in 
Section~\ref{sec:design}. This means that it was relatively easy to 
implementation the database, in accordance with the design. We only had to make 
sure that our translation from the ER diagram to the MySQL EER diagram was 
correct to guarantee that the implemented database worked as desired. However, 
this proved more difficult than expected. Because although it could be directly 
translated, it did happen that the way we actually wanted to implement it, was 
not strictly in accordance with the translation of the design. This sometimes 
resulted in changes of the design, as we realized the design was not in 
accordance with the desired functionality. Other times the desired effect was 
achieved although not in accordance with the translation. This can be seen in 
the relations \emph{from} and \emph{to} in which the entity \emph{Track} has 
total participation. 


\subsection{MySQL EER Diagram}

\begin{figure}[h]
    \centering
    \includegraphics{img/DTU-logo}
    \caption{EER Diagram implemented in MySQL}
    \label{fig:EER}
\end{figure}


\subsection{Functionality} % Other title?

% Specs require: Functions, procedures, transactions, triggers and events
% Make sure to include all of these in the design or possibly in the intro
